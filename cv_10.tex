%%%%%%%%%%%%%%%%%%%%%%%%%%%%%%%%%%%%%%%%%
% Friggeri Resume/CV
% XeLaTeX Template
% Version 1.2 (3/5/15)
%
% This template has been downloaded from:
% http://www.LaTeXTemplates.com
%
% Original author:
% Adrien Friggeri (adrien@friggeri.net)
% https://github.com/afriggeri/CV
%
% License:
% CC BY-NC-SA 3.0 (http://creativecommons.org/licenses/by-nc-sa/3.0/)
%
% Important notes:
% This template needs to be compiled with XeLaTeX and the bibliography, if used,
% needs to be compiled with biber rather than bibtex.
%
%%%%%%%%%%%%%%%%%%%%%%%%%%%%%%%%%%%%%%%%%

\documentclass[]{friggeri-cv} % Add 'print' as an option into the square bracket to remove colors from this template for printing
\addbibresource{bibliography.bib} % Specify the bibliography file to include publications


\begin{document}

\header{robert}{ kuska}{Software Engineer} % Your name and current job title/field

%----------------------------------------------------------------------------------------
%	SIDEBAR SECTION
%----------------------------------------------------------------------------------------

\begin{aside} % In the aside, each new line forces a line break
\section{contact}
Chalupkova 159
Cadca 02201
Slovakia
~
+420 775 446 230
~
\href{mailto:rkuska@gmail.com}{rkuska[at]gmail.com}
\section{languages}
slovak, czech
english fluently
german basics 
\section{programming}
{\color{red} $\varheartsuit$} Python
Java, C, SQL
HTML \& CSS
\end{aside}

%----------------------------------------------------------------------------------------
%	EDUCATION SECTION
%----------------------------------------------------------------------------------------

\section{education}

\begin{entrylist}

%------------------------------------------------

\entry
{2013--now}
{Masters {\normalfont of Informatics}}
{The Masaryk University, Brno}
{\emph{Automatic Text Summarization}}


%------------------------------------------------

\entry
{2009--2013}
{Bachelor {\normalfont of Informatics}}
{The Masaryk University, Brno}
{\emph{Methods of automatic testing in Python language} \\ This thesis deals with the problematics of testing in Python. It contains a description of the usage and code examples of the most common freely available test libraries. The thesis is also a description of the libraries used for creating fake objects. The conclusion contains a summary of the information and comparison of performance of libraries. Selected libraries are applied to the MCC project.}


%------------------------------------------------

\end{entrylist}

%----------------------------------------------------------------------------------------
%	WORK EXPERIENCE SECTION
%----------------------------------------------------------------------------------------

\section{experience}


\begin{entrylist}

%------------------------------------------------

\entry
{2013--Now}
{RED HAT}
{Brno, Czech Republic}
{\emph{Associate Software Engineer} \\
Maintaining RPM packages of python and related libraries in Fedora and RHEL.\\
Developing and maintaining RPM packages of python's related Software Collections (RHSCL).\\
Developing following projects:
\begin{itemize}
\item pyp2rpm - converting PyPI packages into specfiles
\item spec2scl - converting classic specfiles into scl specfiles
\item scltests - framework for testing scl rpm packages
\end{itemize}}

%------------------------------------------------

\entry
{2012--2013}
{EMBEDIT}
{Brno, Czech Republic}
{\emph{Software Tester} \\
Part time job where I assisted with moving from MS Exchange to Lotus Notes.}

\end{entrylist}


%----------------------------------------------------------------------------------------
%	AWARDS SECTION
%----------------------------------------------------------------------------------------

\section{certificates}

\begin{entrylist}

%------------------------------------------------

\entry
{2013}
{Red Hat Certified System Administrator}
{Red Hat}
{Red Hat Enterprise Linux 6 ID:130-174-329}

%------------------------------------------------

\end{entrylist}

%----------------------------------------------------------------------------------------
%	INTERESTS SECTION
%----------------------------------------------------------------------------------------

\section{interests}

\textbf{professional:} machine learning, algorithms, testing, problem-solving \textbf{personal:} reading, photography, audio, cooking, workouts

%----------------------------------------------------------------------------------------
%	PUBLICATIONS SECTION
%----------------------------------------------------------------------------------------

\section{volunteering}

\begin{entrylist}

\entry
{2014}
{DEVCONF}
{Brno}
{Taking care of workshops in specific lab.}

\entry
{2015}
{DEVCONF}
{Brno}
{Taking care of workshops in specific lab.}

\end{entrylist}

%----------------------------------------------------------------------------------------

\end{document}
